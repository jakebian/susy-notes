This note covers standard material in the representation theory of various groups useful in physics.

In contemporary physics literature, the student would turn to any introductory text on SUSY or any of the standard QFT texts covering spinors and be greeted by scary notation involving dotted and undotted indices, they would then be expected to become adept at manipulating these indices and accumulate various mnemonics.

The approach here is to do everything in a index-free fashion as far as possible, and introduce indices only as convenient (and messy) notation for objects which can be defined and studied in an index-free manner. The author feels that the algebraic structure of the objects studied become much clearer in the index-free language.
